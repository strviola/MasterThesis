\documentclass[a4paper,12pt]{jsarticle}
\title{Webアプリケーションにおける不正行為の\\統計的検出}
\author{鈴木竜太}
\date{\today}
% packages
\usepackage[dvipdfm]{graphicx}
\usepackage{listings}
\usepackage{ascmac}
\usepackage{here}
\usepackage{txfonts}
\usepackage{amssymb}

% listings(ソースコード表示スタイル)の設定
\lstset{
	language=Python,
	basicstyle=\ttfamily,
	commentstyle=\textit,
	numbers=left,
	showstringspaces=false,
}

% environment
\newtheorem{pydef}{Method}
\newtheorem{mathdef}{定義}
\newtheorem{lemma}{補題}

% definition
\newcommand{\savefunc}{{\tt save\_to\_outer}}
\newcommand{\param}{\paragraph{パラメータ}}
\newcommand{\pyreturn}{\paragraph{返り値}}
\newcommand{\Eps}{{\it Eps}}
\newcommand{\R}{\mathbb{R}}
\newcommand{\MinPts}{{\it MinPts}}
\newcommand{\N}{\mathbb{N}}
\newcommand{\reach}{\sim_{\Eps, \MinPts}}
\newcommand{\dreach}{\reach'}
\newcommand{\conn}{\simeq_{\Eps, \MinPts}}
\newcommand{\plabel}{.{\rm label}}
\newcommand{\la}{\leftarrow}
\newcommand{\notyet}{{\rm UNCLASSIFIED}}
 % コマンド・パッケージ設定
\begin{document}

\maketitle % 表紙

\newpage
\tableofcontents % 目次

\newpage
\part{はじめに}
\section{背景}
\subsection{不正行為の検出}
ネットワークを用いたアプリケーションは広くアクセス可能であるという利点がある反面,しばしば脆弱性を悪用した攻撃の対象になりうる.
Webアプリケーションの利用の一つとしてオンラインゲームが存在するが,これも攻撃の対象として例外ではない.
オンラインゲームにおける不正行為は特にチートと呼ばれ,チート検出に関して数多くの既存研究が存在する.
過去のチート検出に関する研究のほとんどはMMORPGやFPSを対象としており,チートの内容もゲーム内のキャラクターの移動に関するものである.
このため,過去の研究ではサーバーへの入力のみに注目しており,実際にそれで充分であった.
例えば\cite{botcraft}では代表的なMMORPGである ``World of Warcraft'' における,プログラムによる自動操作キャラクター(Bot)の検出を目的としている.\cite{gamefps}はオンラインFPSにおけるキャラクターの操作経路を取得し,これを正常なプレイヤーの操作経路と比較するというアプローチをとっている.

\subsection{一時接続ゲーム}
近年,ゲーム上の入力の全てではなく,その中の一部だけ,もしくはそれを加工したデータをサーバーに送信する設計のオンラインゲームが普及しつつある.
本論文ではこの特徴を「一時接続」と呼ぶ.
一時接続ゲームは以下の理由から,特に携帯端末プラットフォームにおいて顕著である.

\begin{itemize}
\item
消費電力を抑える.携帯端末はバッテリーが有限であり,さらにゲーム以外のアプリケーションにも影響するため.
\item
屋外での利用も想定すると,電波環境が安定しない.
\end{itemize}

一時接続ゲームにもチートは存在しうる.より一般的に,以下の条件を満たすサーバー・クライアント型アプリケーションにはチートが発生する.

\begin{enumerate}
\item
サーバーにデータを送信し,他の人(アカウント)と共有する
\item
クライアントの設計上、送信データに「制約」が存在する
\item
制約外のデータ(「チート」と呼ぶ)を送信することで、他人の行動や意思決定に影響する
\item
チートを行うことで自分に有利になる
\end{enumerate}

ここで,この条件が妥当である理由を,具体例を交えて説明する.
まず1は最大の前提である.
そもそもインターネットに繋がっていないゲームであれば,たとえ改造者がクライアントを改造するなどしてプレイヤーに不正に有利な状態を作ったとしても何ら問題はない.
それは改造者が勝手に遊んでいるだけで,他人に迷惑をかけていない.
\footnote{二次創作物やそれに対する著作権などの問題が発生することが考えられるが,今回の研究の興味の対象ではない.}
また2に関して,「制約」が存在する全てのアプリケーションでチートが問題になるとは限らない.
例えばTwitterは「140文字以上の文字列を送信できない」という「制約」が存在するが,この制約を守らないチート的な送信があったところでそれは他人に何ら影響はない.

問題は3の「他人に影響する」という特徴である.
オンラインゲームは遠い所にいる他人との同時プレイを可能にするものであり,それが醍醐味のひとつでもある.
一方でそれを可能にしている技術には欠陥が含まれることも多く,チートを許してしまうケースも存在する.
例えば,ガンホー社の「パズル\&ドラゴンズ」(パズドラ)では,外部ツール\cite{ghostrouter}を利用してゲーム内のデータを任意の数値に書き換えるチートが可能である.
このチートを行った時,プレイヤー本人が有利な状態を不正に作る
\footnote{強力なキャラクターを使用できる,既存のキャラクターの能力を向上させる,などが可能である.}
だけでなく,他人の意思決定に影響してゲーム内ポイントをより多く獲得することも期待できる.

\section{問題点}
\subsection{詐称チート}
一時接続ゲームにおけるチートの多くは,様々なパラメータを偽り,チート利用者に有利な状態を作ることを目的とする.
この操作は,ゲーム内のプロセスを経ることなく,キャラクターやアイテムを使用可能にし,またキャラクターの性能を上げるなどの効果をもたらす.
本研究ではこのチートに注目し,「詐称チート」と定義する.
既存のチート検出システムは,詐称チートを検出できるかどうか自明ではない.これはサーバーへの入力(または入力の集合)が同一であっても,チートと見なすべき場合と正常と見なすべき場合が存在するためである.

具体的な例で考える.RPGにおいて,アイテムAはゲーム終盤のダンジョンXでのみ手に入るという環境を想定し,「プレイヤーPがAを取得した」というリクエストRがサーバーに送信されたとする.
もしRが正常なリクエストであれば,Pは実際にゲーム内のイベントを経由し,ダンジョンXに到達したはずである.
一方Rが詐称チートであれば,Pはゲームのプログラムを不正に加工し,リクエストのみを送信したことになる.
ここで,チートの判定のために単純な条件式を用いることは得策ではない.
「PはXに入るためのイベントYを達成し,アイテムZを持っていて,経験値がN以上であり…」と,ゲームの設計によってその判定式・フラグはいくらでも複雑になりうる.

\section{研究の目的}
本研究では,一時接続ゲームにおけるクライアント設計に非依存な詐称チートの検出を目的とする.すなわち,ゲームの設計を含むクライアントの情報を得られない環境で,詐称チートをサーバー上で検出する方法を提案する.

\section{アプローチ}
\subsection{成長過程座標}
本研究では,ユーザーごとの「ゲームの進行状態」に注目した.これはユーザーが自分で操作できない数値である.また,ゲームの進行状態に影響されるリクエストパラメータを「成長パラメータ」と定義し,ゲーム進行状態と成長パラメータを連結した数値リストを「成長過程座標」としてこれに注目した.

成長過程座標には,詐称チートと正常入力の差が生じることが期待できる.例えば先のRPGの例において,ゲームの進行状態としてゲームのプレイ時間$t$,成長パラメータとしてアイテムのID$i$をとれば,成長過程座標$(t, i)$が作成できる.ここで,一般的なプレイヤーはアイテムAの取得に50時間かかるとする.すなわち多くのは$(50, A)$前後である.この時,プレイヤーPから作成された成長過程座標が$(10, A)$であれば,Pはチートを行っている可能性が高い.

\subsection{クラスタリングによる詐称チートの検出}
成長過程座標に関しても,何がチートに該当するかはゲームの設計に依存しており,単純な条件式で判別すべきでない.本研究ではクラスタリングを用いることで,ゲームの設計に非依存な判定が可能であることを示す.すなわち,
大多数の正常な入力を「正常値クラスタ」,一部のチート入力を「ノイズ」として判定することで,詐称チートを判定する.既に複数のクラスタリングアルゴリズムが知られているが,本研究ではその中のDBSCANを採用した.DBSCANはクラスタ数を事前に見積もる必要がなく,またどのクラスタにも属さないノイズを検出できるため,詐称チートの検出に適していると考えたためである.

\section{実験}
\subsection{ちくわ栽培キット}
このアプローチを検証するために,実際のオンラインゲームにおける多数の正常値入力セットを用意する必要がある.本研究では株式会社アドバンスドテクノロジー様の協力により,Android向けソーシャルゲーム「ちくわ栽培キット」のサーバーアクセスログの提供を受けた.

\subsection{結果}
提供された2128件の正常値サンプルと自分で用意したチートセットを利用し,DBSCANを用いたチート判定を行った.その結果,チートの数が正常値の数と比べて充分に少なく,かつチートによりチート利用者が極端に有利になる場合においては,クラスタリングがチート判定方法として有効であると示した.

\section{貢献}
既存のチート検出器では対応できないチートとして,一時接続ゲームにおける詐称チートを定義した.詐称チートに対して,ゲーム進行状態と成長パラメータを組み合わせた成長過程座標に対するクラスタリングが検出方法として有効であることを,実験を通して示した.


\section{関連研究}
本研究はオンラインゲームのチート検出を目標としているため,まずこの分野の先行研究について述べる.次に,機械学習を用いた不正入力の検出に関連して侵入検知システム(Intrusion Detection System, IDS)を取り上げる.

\subsection{オンラインゲームのチート検出}
Gianveccioら\cite{botcraft}はMMOGのひとつである「World of Warcraft」におけるBot(自動操縦プログラム)を観測的に検出する方法を提案している.この研究は,操作キャラクターのキーボードおよびマウスの操作をリアルタイムに観測し,異常に正確もしくは素早い操作を行っているキャラクターを検出することで,不正行為を発見する手法を提案した.Laurensら\cite{gamefps}はチートを許してしまう技術的もしくは開発プロセス上の欠陥について調査ている.また,それらに対する防御策として,オンラインFPSにおけるWall-hack cheatingを観測的に検出するシステムを提案している.

Shiら\cite{gamep2p}はP2Pアーキテクチャを採用したMMOGにおいて,各プレイヤーの操作から数値的に信頼度を計算することでチート検出を試みている.ここでいう「信頼度」の計算には,同様にプレイヤーの逐次的な行動(キャラクターの動く速度,その正確さ,など)が必要であるため,一部のデータのみをサーバー経由で共有する一時接続ゲームには適用できない. 

\subsection{IDS}
Portnoyら\cite{cluster}はTCPアクセスログにクラスタリングを用いることで,DOS攻撃やR2L攻撃の教師なし侵入検知を行った.Kruegelら\cite{httplearn}はHTTPリクエストパラメータに対して,複数の統計処理や自然言語解析に基づいた解析手法を適用することで,バッファオーバーフローやXSS(Cross Site Scripting)などの侵入検知を行った.
Leeら\cite{mining}はデータマイニングアルゴリズムであるRIPPER\cite{ripper}を用いることで,データを解析し拡張性の高い侵入検知システムを構築する研究を行っている.
 % 「はじめに」

\newpage
\part{方法}
この章では,今回の実験の対象としたゲーム「ちくわ栽培キット」と,「ちくわ」におけるチートおよびそれの検出方法について述べる.

\section{ちくわ栽培キット}
\subsection{概要}
「ちくわ栽培キット」は,「ちくわ」をメインキャラクターとした育成シミュレーションゲームである.このゲームは「栽培」「設備投資」「バトル」の要素からなる\footnote{この他に,収穫したちくわの詳細情報を閲覧できる「図鑑」機能がある.}.

\subsubsection{用語の解説}
\paragraph{ちくわ}
ゲーム中で取得可能なキャラクターである.このゲームは「ちくわ」を収集することを最大の目的とする.60種類\footnote{2014年1月現在.今後増える可能性がある.}存在し,それぞれ特徴が異なる.取得したちくわは一時的に「収穫箱」に保存される.

\paragraph{収穫箱}
取得したちくわを保存する機能.収穫箱に存在するちくわに対して,さらに操作が可能である.

\paragraph{ちくわポイント(TP)}
ゲーム内通貨.

\subsubsection{操作の説明}
\paragraph{栽培・収穫}
ちくわを生成する.出現したちくわは「収穫」することで収穫箱に保存される.

\paragraph{売却}
収穫箱にあるちくわを,ちくわごとに定義されたTPと交換する.

\paragraph{設備投資}
TPを消費し,発生するちくわの種類など,栽培に関するパラメータを変更(向上)できる.

\paragraph{バトル}
他人と対戦する.その際,以下の手続きを経る.

\begin{enumerate}
\item
「対戦」コマンドを選択する.この際サーバーとの通信が発生し,収穫箱にあるちくわの種類(ID: 整数)と数(整数)が全て送信される.
\item
プレイヤーは自分が操作するちくわを選択する.「自軍」と呼ぶ.
\item
サーバーに保存されているちくわのリスト(所持者のユーザーID,ちくわID,ちくわの数)から一部を取得する.
\item
プレイヤーはリストから対戦相手とするちくわを選択する.「敵軍」と呼ぶ.
\item
バトルが開始する.プレイヤーは一定時間以内画面をタップすれば勝ち,出来なければ負けである.その時間は自軍と敵軍のパラメータから計算される.
\item
勝敗に応じてイベントが発生する.
\begin{itemize}
\item
プレイヤーが勝った場合,敵軍のちくわのコピーがプレイヤーの収穫箱に追加される.この時,敵軍のちくわIDと数がサーバーに送信される.
\item
プレイヤーが負けた場合,自軍のちくわが収穫箱から削除される.この時,敵軍のちくわの所持者に報酬が送信される.報酬は敵軍のちくわの売却TPと同一である.
\end{itemize}
\end{enumerate}


\subsection{実装}
\subsubsection{クライアントサイド}
クライアントサイドの実装はAndroidで行った.バトル機能を利用する際にHTTP通信を利用する.

\subsubsection{サーバーサイド}
サーバーサイドの実装にはDjango\cite{django}を用いた.DjangoはPython\cite{python}で実装されたWebアプリケーションフレームワークである.Djangoを用いたWebアプリケーション開発では,以下の要素を定義する必要がある\footnote{クライアントがWebアプリケーションである場合,この他にテンプレートHTMLを定義する.今回の「ちくわ栽培キット」のクライアントはAndroidであるため利用していない.}.

\begin{itemize}
\item
モデル.オブジェクト指向言語におけるクラス,データベースにおけるテーブルに該当する.
\item
ビュー関数.リクエストオブジェクトを受け取り,データベースを利用した操作を行い,レスポンスオブジェクトを返す.
\item
URLディスパッチャ.URLとビュー関数を対応づける.
\end{itemize}

\subsection{このゲームで考えられるチート}
\paragraph{過剰ちくわID}
ゲームの進行状態によって,取得可能なちくわIDには制限が存在する.制限を解放するには,設備投資を行って発生ちくわIDの上限を上げる必要がある.しかし,不正アクセスにより制限外のIDをサーバーに送信することが可能である.

\paragraph{過剰ちくわ数}
収穫箱の容量には上限が存在する\footnote{2014年1月現在,この上限は200で固定である.ゲームに課金することでこの上限を拡張する機能を実装予定である.}.従ってサーバーに送信される「ちくわ数」にも制限が存在する.しかし,ここで不正アクセスにより極端に多いちくわ数を送信することも可能である.

\paragraph{過剰報酬}
プレイヤーがバトルに負けた場合,敵軍の所持者に報酬としてTPが送信される.この報酬TPは敵軍のちくわのIDと数によって決まる.しかし,不正アクセスを行うと任意の数を送信することが可能である.

 % 「ちくわ栽培キット」ルール説明
\section{チートの定義}
この節では,本研究の興味の対象とするチートの定式化と,それを検出するための概念の定義を行う.

\subsection{詐称チート}
一般に,オンラインゲームにおいて既に多種多様のチートが報告されている\cite{cheatclass}.本研究では,その中でも「間違った信用を利用したチート」,すなわちクライアントサイドのプログラムを不正に改造することによるチートを興味の対象とする.

「ちくわ」の例では,クライアントアプリケーションのソースコードの漏洩
\footnote{もちろん,あまり簡単にサーバーに侵入できないような対策をとっているが,いずれも本質的にアプリケーション以外からのアクセスを防ぐものではない.}
などにより,ゲームの状態を無視した不正なHTTPリクエストをサーバーに送信することが可能である.本研究ではこの概念を定式化した{\bf 詐称入力}および{\bf 詐称チート}を以下の通り定義する.

\begin{mathdef}[詐称入力]
サーバー・クライアント型アプリケーションで,クライアントアプリケーションのソースコード以外に由来するサーバーへのHTTPリクエストを詐称入力と呼ぶ.
\end{mathdef}

\begin{mathdef}[詐称チート]
詐称入力のうち,クライアントアプリケーションが以下の条件を満たす時,それを詐称チートと呼ぶ.

\begin{enumerate}
\item
クライアントの設計上、送信データに「制約」が存在する.
\item
詐称入力を送信することで、他人の行動や意思決定に影響する.
\item
詐称入力を行うことで自分に有利になる.
\end{enumerate}

\end{mathdef}

\subsection{成長過程座標を用いた詐称チートの検出}
本研究では,「ちくわ」を例に取って詐称チートを検出することを試みる.詐称チートは以下の理由から,単純な条件で判定できるかどうか自明ではない.

\begin{enumerate}
\item
サーバーからは,詐称入力か通常入力かどうか判別できない.
\item
送信されうるHTTPリクエストパラメータはクライアントアプリケーションの設計に依存する.パラメータの間に依存関係が存在する場合もありうる.
\item \label{cheatnot}
同じHTTPリクエストパラメータでも,詐称チートになる場合とそうでない場合が存在する.
\end{enumerate}

特に理由\ref{cheatnot}はこの問題を難しくしている.例として,「ちくわ」はバトルの際に収穫箱の情報をサーバーに送信する.この時のHTTPリクエストパラメータは「ちくわ毎に割り当てられたID」と「そのIDのちくわのその時点での所持数」である.ここで,各時点における取得可能なちくわ(ID)には制限が存在し,その制限は設備投資を行うことで拡張できる.このため,設備制限外の入力は詐称チートと見なすのが自然であるが,設備制限の情報をサーバーが取得するのはコストがかかるため得策ではない
\footnote{仮にサーバー側で設備制限情報を用いた判定メソッドを実装する場合,同じ動作をするソースコードをサーバーとクライアントで2重に用意する必要がある.また,その判定メソッドは汎用性がなく,ビュー関数ごとに用意しなければならない}.

\paragraph{アプローチ: 成長過程座標}
詐称チートが発生するのは,そのパラメータにアプリケーションの設計上の制約が存在するためである.本研究では以下の条件を満たすパラメータを特に{\bf 成長パラメータ}として区別する.

\begin{enumerate}
\item
ゲームの設計上,パラメータが取りうる値に制約が存在する.
\item
ゲームの進行に従って,その制約が拡張される.
\end{enumerate}

成長パラメータと,一般的に取得可能なゲームの進行状態を組み合わせれば,詐称チートが検出できることが期待できる.本研究ではこの目的を達成するために{\bf 成長過程座標}を定義し,ゲームの進行状態として{\bf ゲーム歴}を採用した.

\begin{mathdef}[成長過程座標]
ゲームの進行状態を表す数値(複数可)を$p_1, \dots, p_m (m \geq 1)$,成長パラメータを$q_1, \dots, q_n (n \geq 1)$とする.この時,ベクトル \[
(p_1, \dots, p_m, q_1, \dots, q_n)
\] を成長過程座標とする.
\end{mathdef}

\begin{mathdef}[ゲーム歴]
ユーザーがHTTPリクエストを送信した日時とユーザーがWebサービスに会員登録した日時の差の秒数をゲーム歴とする.
\end{mathdef}
 % 詐称チートと成長過程座標の説明
\section{サーバーへのアクセスログの取得}
本研究では,詐称チートを検出するために「ゲーム進行状態」と「成長パラメータ」を組み合わせた「成長過程座標」を用いる.
そのために,必要な情報をサーバーへのHTTPアクセスから取得する必要がある.
この節では,HTTPアクセスを蓄積し,そこから詐称チートの検出に必要な情報を取得する方法を述べる.

\subsection{実装の必要性}
Djangoのビュー関数はHTTPリクエストオブジェクトを受け取り,リクエストからパラメータを取り出したあと,データベース処理などを行ってレスポンスを返す.
しかし,受け取ったリクエストオブジェクト自体は破棄される.
一方,Apache ServerにはCommon Log Format\cite{clf} (CLF) 形式でログを保存する機能が存在する.
しかし,CLFでアクセス元を区別する方法はIPアドレスのみであり,アクセス元に付随するアプリケーション上の情報は取得できない.
従って,この情報ではチートの検出に不十分である.

\subsection{リクエストオブジェクトとログ配列}
本研究では,PythonのPickleモジュール\cite{pickle}を用い,リクエストオブジェクトを直接外部ファイルに保存する(Pickle化する)手法をとった.
ただし,Pickleモジュールには制限があり,Pickle化できないオブジェクトが存在する.不幸なことに,実際のリクエストオブジェクトにはPickle化できないオブジェクトが含まれるため,そのままではPickleを適用できない.本研究では,成長過程座標に必要な情報を損なわない程度にリクエストオブジェクトから情報を抜き出し,この制限を回避している.

以後,保存されたリクエストオブジェクトを「ログ配列」,ログ配列が保存されている外部ファイルを「ログファイル」と呼ぶ.
リクエストオブジェクトには以下の情報が含まれる.

\begin{itemize}
\item
ユーザープロフィール.
\begin{itemize}
\item
ユーザーID(Primary Key, PK).会員登録時にサーバーが自動的に付与する.
\item
メールアドレス.
\item
表示名.
\item
最終ログイン日時.
\item
会員登録日時.
\end{itemize}
\item
リクエストパラメタ.Pythonの辞書(Dictionary)形式.
\end{itemize}

\subsection{ログ配列へのリクエストオブジェクトの追加}
リクエストオブジェクトを外部ファイルに保存するためのメソッドを \savefunc として定義した.この定義は以下の通りである.

\begin{pydef}
\savefunc{\tt (request, filename)}

予め保存されているログファイルを開き(これに失敗した場合は新規作成),ログファイルからログ配列を生成し,ログ配列にリクエストオブジェクトを付け足し,ログファイルに保存する.

\param
\begin{itemize}
\item
{\tt request} 保存するリクエストオブジェクト.
\item
{\tt filename} ログファイルのファイル名.
\end{itemize}

\end{pydef}

リクエストオブジェクトは,Djangoのビュー関数おいて必ず存在する.
よってこの関数はDjangoアプリケーションであればモデルやビュー関数の設計に関わらず適用することができる.
この外部ファイルは``unpickle''を適用することでPythonインスタンスとして扱うことができる.

\paragraph{使用例}
\savefunc はビュー関数中に挿入する形で用いる.以下に実際のコード例を示す.13行目に \savefunc を挿入し,リクエストオブジェクトをログファイルに追加する操作を行っている.

\begin{lstlisting}
def tikuwa_update(request):
    '''Update the request.user's TikuwaBox to latest state.
    Tikuwa Box is given as JSON text.'''

    if request.method == 'POST':
        try:
            tid = request.POST['tid']
            hasnum = request.POST['hasnum']
        except KeyError:
            return responses.bad_request(message='Field lacks')

        # save logs and return JSON ACK
        request_logging.save_to_outer(request, 'tikuwa_update')
        return backends.tikuwa_sync(request.user, tid, hasnum)

    else:
        responses.bad_request(message='Not POST request')
\end{lstlisting}

\subsection{ログ配列から座標への変換}
本研究では,ログ配列から詐称チートを検出するために空間的クラスタリングを用いる.
そのために,リクエストオブジェクトから必要な情報を抜き出し,成長過程座標を作成する必要がある.
以下がその関数である.

\begin{pydef}
{\tt make\_point(request, *keys)}
単一のリクエストオブジェクトを成長過程座標に変換する.

\param
\begin{itemize}
\item
{\tt request} 変換対象のリクエストオブジェクト.
\item
{\tt keys} リクエストオブジェクトに含まれる,成長パラメータとして用いるパラメータのキーの名前.
\end{itemize}

\pyreturn
リスト形式による成長過程座標.最初の項はゲーム歴,2項目以降は指定したパラメータである.

\end{pydef}

例を用いて説明する.以下のリクエストオブジェクトを成長過程配列に変換する.このオブジェクトを保存する変数を{\tt my\_request}とする.

\begin{itemize}
\item
ユーザープロフィール
\begin{itemize}
\item
会員登録日時: 2014年01月01日 00:00:00
\item
最終ログイン日時: 2014年01月02日 12:34:56
\end{itemize}
\item
リクエストパラメータ
\begin{itemize}
\item
{\tt "tid": 0}
\item
{\tt "num": 24}
\end{itemize}
\end{itemize}

以下のコードは,対話的に{\tt my\_request}を成長過程座標に変換し,変数{\tt my\_point}に代入する方法を示す.

\begin{lstlisting}
>>> my_point = make_point(my_request, 'tid', 'num')
>>> my_point
[131696, 0, 24]
\end{lstlisting}

まず進行状態としてゲーム歴を取得する.
ログ配列のユーザー情報には「会員登録日時」と「最終ログイン日時」が含まれるので,その差を取ればゲーム歴が取得できる.
成長パラメータはPython辞書オブジェクトなので,キー名を文字列で指定することで取得できる.
 % アクセスログから成長過程座標に変換するまで
\section{詐称チートの検出}
この節では,成長過程座標の列から詐称チートを検出するために用いたクラスタリングアルゴリズムであるDBSCAN (Density Based Spacial Clustering of Application with Noise \cite{dbscan})について述べる.

\subsection{DBSCANの概要}
DBSCANは直感的に,ユークリッド空間上の近い(距離が短い)点を同一のクラスタと見なし,どのクラスタからも遠い(距離が長い)点をノイズと見なすことでクラスタリングを行うアルゴリズムである.従って,以下の2つのパラメータが重要である.

\begin{itemize}
\item
実数$\Eps$.2点間の距離が$\Eps$以内であれば同一のクラスタと見なす.
\item
自然数$\MinPts$.クラスタに含まれる点の数が$\MinPts$未満であれば,そのクラスタはノイズと見なす.
\end{itemize}

\subsection{定義と準備}
成長過程座標を$k$次元ユークリッド空間$D$上の点と見なし,個々の成長過程座標を$p,q \in D$とする.ここで$D$上には距離関数$d(p,q)$が定義できるものとする.

\begin{mathdef}[$\Eps$近傍]
点$p \in D$,距離$\Eps \in \R$に対して,点$p$の$\Eps$近傍$N_{\Eps}(p)$を以下の式で定義する.
\[
	N_\Eps(p) = \{q \in D \mid d(p,q) \leq \Eps\}
\]
\end{mathdef}

\begin{mathdef}[直接到達可能性]
点$p,q$が以下の条件を満たす時,$p$は$q$から$\Eps, \MinPts$に関して直接到達可能であると呼び,$p \dreach q$と表す.
\begin{enumerate}
\item
$p \in N_\Eps(q)$
\item
$|N_\Eps(q)| \geq \MinPts$
\end{enumerate}

ただし,集合$S$に対して$S$の要素の個数を$|S|$と表す.
\end{mathdef}

\begin{mathdef}[到達可能性]
以下の条件を満たす点列$p_1, \dots, p_n (n \in \N)$が存在する時,$p$は$q$から$\Eps, \MinPts$に関して到達可能であると呼び,$p \reach q$と表す.

\begin{enumerate}
\item
$p_1 = p \cap p_n = q$
\item
$1 \leq i \leq n-1$を満たす全ての$i \in \N$について,$p_i \dreach p_{i+1}$である.
\end{enumerate}
\end{mathdef}

到達可能性は一見対称な条件に見える.しかし点$p$の周辺に点が少ない時,直接到達可能性の第2条件$|N_\Eps(p)| \geq \MinPts$が満たされない場合が存在するため,対称ではない.
この時,点$p$はクラスタの境界点であると呼ぶ.
言い換えると,$p$が境界点である時,$p \reach q$であっても$q \reach p$であるとは限らない.

境界点どうしの関係を述べるため,以下の性質を定義する.

\begin{mathdef}[接続性]
ある点$r$が存在して,$r \reach p$かつ$r \reach q$である時,$p$と$q$は$\Eps, \MinPts$に関して接続していると呼び,$p \conn q$と表す.
\end{mathdef}

これらの定義を用いて,クラスタおよびノイズを次の通り定義する.

\begin{mathdef}[クラスタ]
クラスタ$C$は以下の条件を満たす$D$の空でない部分集合である.

\begin{enumerate}
\item
$\forall p, q$について,もし$p \in C$かつ$q \reach p$であれば$q \in C$である.
\item
$\forall p, q \in C$について,$p \conn q$である.
\end{enumerate}
\end{mathdef}

\begin{mathdef}[ノイズ]
$C_1, \dots, C_n (n \in \N)$を$D$のクラスタであるとする.この時,全ての$1 \leq i \leq k$について点$p$が$p \not \in C_i$である時,点$p$はノイズである.
\end{mathdef}

以下の補題は,次に述べるDBSCANアルゴリズムの正当性を示すために重要である.

\begin{lemma}
点$p$を$D$上の点とし,$|N_\Eps(p)| \geq \MinPts$であるとする.この時,集合$O = \{o \mid o \in D \cap o \reach p\}$はクラスタである.
\end{lemma}

\begin{lemma}
$C$をクラスタとし,点$p$を$p \in C$かつ$|N_\Eps(p)| \geq \MinPts$である点とする.この時,$C$は集合$O = \{o \mid o \reach p\}$と等しい.
\end{lemma}

\subsection{DBSCANアルゴリズム}
以上の定義を用いて,DBSCANアルゴリズムを以下の通り定義する.

\paragraph{入力}
\begin{itemize}
\item
$D$: 成長過程座標の列.
\item
$\Eps$: 実数.
\item
$\MinPts$: 自然数.
\end{itemize}

\paragraph{初期化}
\begin{itemize}
\item
全ての$p \in D$について,$p\plabel = \notyet$
\item
$l = 0$
\end{itemize}

ここで,$p\plabel$は$p$が所属するクラスタを表すラベルであり,ノイズを$-1$で表す.初期状態では$p$は全て未分類(unclassified)である.$l$はその時点でのクラスタの数である.

\paragraph{アルゴリズム}
\begin{enumerate}

\item \label{choose}
$D$から点をひとつ選び$p$とする.$D \la D \setminus \{p\}$.

\item
集合$S$を$S \la N_\Eps(p)$とする.

\item
$|S| < \MinPts$であれば\ref{choose}に戻る.そうでなければ\ref{reach}に進む.

\item \label{reach}
$p\plabel \la l$.

\item
$S \la S \setminus \{p\}$.

\item \label{seeds}
$S$から点をひとつ選び$p_1$とする.

\item
集合$S_1$を$S_1 = N_\Eps(p_1)$とする.

\item
$|S_1| \geq \MinPts$であれば\ref{search}に進む.そうでなければ\ref{endsearch}に進む.

\item \label{search}
$S_1$から点をひとつ選び$p_2$とする.$S_1 \la S_1 \setminus \{p_2\}$.

\item
$p_2\plabel = \notyet$であれば$S \la S \cup \{p_2\}$.

\item
$p_2\plabel \la l$.

\item
$S_1 = \{\}$であれば\ref{endsearch}に進む.そうでなければ\ref{search}に戻る.

\item \label{endsearch}
$S \la S \setminus \{p_1\}$.

\item
$S = \{\}$であれば\ref{labelchange}に進む.そうでなければ\ref{seeds}に戻る.

\item \label{labelchange}
$l \la l + 1$.

\item
$D = \{\}$であれば終了.そうでなければ\ref{choose}に戻る.

\end{enumerate}

\subsection{詐称チート検出におけるDBSCAN}
このアルゴリズムを採用した理由は,DBSCANのもつ以下の特徴による.

\begin{itemize}
\item
クラスタ数を事前に指定する必要がない.ゲームの成長過程座標は,個人の遊び方により複数のクラスタを形成する(例えば,ヘビーユーザーとライトユーザー)ことが考えられるため,事前にクラスタ数を見積もることが難しい,この点でDBSCANは有利である.
\item
どのクラスタにも属さない座標をノイズとして検出できる.詐称チートは他の成長過程座標から距離的に離れている可能性が高いため,ノイズとして検出されることが期待できる.
\end{itemize}
 % DBSCANの説明 (引用)

\newpage
\part{実験}
この章では,実際に「ちくわ栽培キット」を運用して行った提案手法の実証実験と,その結果について述べる.

\section{実験環境}
\subsection{プログラム動作環境}
\subsubsection{クライアントサイド}
\begin{itemize}
\item
プラットフォーム: Android 3.0 (API11 HONEYCOMB) $\sim$ 4.4 (API19 KITKAT)
\item
端末: ICONIA Tab, Galaxy Nexus, Arrows Z, MEDIAS Tab, 等.
\end{itemize}

\subsubsection{サーバーサイド}
\begin{itemize}
\item
OS: Ubuntu
\item
サーバーソフトウェア: Apache Server
\end{itemize}

\subsection{正常入力の収集}
今回は「ちくわ栽培キット」における「収穫箱の情報をアップロードする」メソッドに注目した.
この実験における成長過程座標は以下の通りである.

\begin{center}
{\tt [time, tid, num]}
\end{center}

ただし,

\begin{itemize}
\item
{\tt time}: 「ちくわ栽培キット」のゲーム暦
\item
{\tt tid}: 送信されたちくわのID.
\item
{\tt num}: IDに対応するちくわの所持数.
\end{itemize}

である.なお,{\tt tid, num}共にゲームの進行上の制限を受ける成長パラメータである.

このメソッドのサンプルを収集するため,
学内および株式会社アドバンスドテクノロジーの従業員に協力を呼びかけ,計10人に「ちくわ栽培キット」を遊んでもらった.この結果,2013年12月24日から同月30日の期間に2128件の正常入力サンプルを集めることができた.

\subsection{予備実験: 正常入力のクラスタ解析}
チートを判定する前に,正常入力がノイズのないクラスタと判定されるかどうか,またその時のパラメータ$\Eps, \MinPts$の値はいくらか確かめる必要がある.なお,正確な外れ値判定を行うためには,$\Eps$はなるべく小さく,$\MinPts$はなるべく大きい方が望ましい.

\paragraph{$\MinPts$の影響}
$\Eps = 0.5$に固定し,$\MinPts$を次々に変化\footnote{$\MinPts = 1$だとノイズが判定できないため意味が無い.}させた時のクラスタ数(\#C),ノイズ数(\#N)は次の通りである.なお詳しくは後述するが,$\Eps = 0.5$はかなり厳しい(ノイズが発生しやすい)条件である.

\begin{table}[htbp]
\begin{center}
\begin{tabular}{c|cc}
$\MinPts$ & \#C & \#N \\ \hline
2 & 13 & 0 \\
3〜4 & 12 & 4 \\
5〜10 & 11 & 8 \\
11 & 10 & 27
\end{tabular}
\end{center}
\end{table}

この結果から,正常値サンプルのうち4点は他の位置から離れたクラスタを形成し,その他の4点がさらに離れた位置で別のクラスタを形成していると推測できる.以後の実験では,$\MinPts=2, 4, 10$の時に注目する.

\paragraph{$\Eps$の影響}
入力を正常サンプルのみとし,$\MinPts = 10$に固定して$\Eps$を変化させた時のクラスタリング結果は以下の通りである.

\begin{table}[htbp]
\begin{center}
\begin{tabular}{c|cc}
$\Eps$ & \#C & \#N \\ \hline
0.4 & 14 & 53 \\
0.5 & 10 & 27 \\
0.6 & 8 & 8 \\
0.7〜0.9 & 5 & 8 \\
1.0〜1.4 & 5 & 2 \\
1.5 & 3 & 0
\end{tabular}
\end{center}
\end{table}

この結果からも,サンプルのうち8点は他のクラスタから離れていることが推測できる.また,$\Eps=0.6, 1.0, 1.5$の時にそれぞれ大きな変化があることも分かる.以後の実験では,この3値に注目する.

\subsection{チートの挿入}
この実験では成長パラメータが2個存在するため,その2個に関する格子状チートセットを用意した.また,解析アルゴリズムであるDBSCANは点と点の間隔に強く影響されるため,格子の間隔は以下の4通りとした.なお,「ゲーム初心者がチートを使ってゲーム終盤でしか手に入らないアイテムを入手した」場合を想定しているため,挿入したチートのゲーム歴は正常入力のそれに比べて無視できる値である.このため,以降チートセットのゲーム歴は0とし,表記を省略する.

\begin{table}[htbp]
\begin{center}
\begin{tabular}{c|ccc}
& 総チート数 & {\tt tid}間隔 & {\tt num}間隔 \\
\hline
Sparse & 15 & 14 & 5000 \\
Mid & 45 & 7 & 1000 \\
Densely & 200 & 3 & 100 \\
Hyper-densely & 200 & 1 & 10
\end{tabular}
\end{center}
\end{table}

\section{結果}
\subsection{攻撃の分布の影響}
予備実験の結果を踏まえ,$\Eps = 0.6, \MinPts = 10$として各チートセットを混ぜたサンプルにDBSCAN解析を行い,False Positive (\#FP), False Negative (\#FN)を測定した.結果は以下の通りである.

\begin{table}[htbp]
\begin{center}
\begin{tabular}{c|ccc}
チートセット & \#C & \#FP & \#FN (rate) \\ \hline
Sparse & 6 & 0 & 1 (6.7\%) \\
Mid & 6 & 0 & 2 (4.4\%) \\
Densely & 5 & 0 & 11 (5.5\%) \\
Hyper-densely & 6 & 8 & 200 (100\%)
\end{tabular}
\end{center}
\end{table}

この結果から,チートが少ない場合はDBSCANによりチートが検出できることが分かる.ただし,チートそのものの密度が正常入力と区別できなくなると,この方法は使えなくなる.

なお,各チートセットで検出できなかった(False Negativeであった)チート入力は以下の通り.

\begin{table}
\begin{center}
\begin{tabular}{l|l}
Sparse & $[0, 1]$ \\ \hline
Mid & $[0, 0], [0, 7]$ \\ \hline
Densely & $[0, 0], [0, 100], [0, 200], [3, 0], [3, 100], [3, 200],$ \\
& $[6, 0], [6, 100], [9, 0], [9, 100], [12, 0]$
\end{tabular}
\end{center}
\end{table}

ゲーム歴が0であることを考えると,この状態で自然な入力は{\tt tid, num}共に小さいものである.従って,DBSCANも自然な入力に近いものに対して弱いといえる.

\subsection{アルゴリズムのパラメータの影響}
実際のゲームの運用環境においては,チートの数は正常入力の数より充分小さいと推測される.すなわち,これらのチートセットのうち最も現実に即しているのはSparseである.
これを踏まえ,Sparseチートセットに対して$\MinPts=10$とし$\Eps$を変更しながら解析を行った.結果は以下の通りである.

\begin{table}[htbp]
\begin{center}
\begin{tabular}{c|ccc}
$\Eps$ & \#C & \#FP & \#FN \\ \hline
0.1 & 34 & 34 & 1 \\
0.2 & 25 & 12 & 1 \\
0.3 & 6 & 0 & 1 \\
$\vdots$ & \multicolumn{3}{c}{$\vdots$} \\
2.2 & 2 & 0 & 1 \\
2.3 & 2 & 0 & 2
\end{tabular}
\end{center}
\end{table}

$\Eps=2.3$の時点でFalse Negativeが増加した.この時検出できなかったチートサンプルは$[14, 1]$である.
予備実験において$\Eps=1.5$でノイズが出なかったことを踏まえると,正常値セットに対してノイズが出ない$\Eps, \MinPts$の値はチート検出にも有効であると結論できる.

\subsection{考察}
\subsubsection{検出失敗の理由}
この実験では,どのチートセットに対してもFalse Negativeが発生した.これは,チートの密度が低い場合は正常入力と近い値が検出できなかったことに起因する.チート利用者にとって,正常入力に近い入力をわざわざチート経由で行う利点はない\footnote{Hyper-denselyチートセットに匹敵する量のチートが送信された場合はこの限りではない.ただし,オンラインゲームの実運用でこれが発生するようなら,サーバーアプリケーションのセキュリティを根本的に見直した方がいいだろう.}.従って,DBSCANは詐称チート検出に効果があると結論できる.

\subsubsection{パラメータの見積もり}
$\MinPts$を小さくすると,チート同士でクラスタを形成し,ノイズと判定されなくなる可能性が高くなる.また$\Eps$を小さくすると逆に正常な入力がノイズと判定される確率が上がる.これらの値は正常値データセット(もしくはなるべく正常値の割合が高いデータセット)に対して予備実験を行い,ノイズが出ない値に調整する必要がある.

\paragraph{チートとノイズの関係}
この実験で,予備実験では正常入力であるにも関わらずノイズと判定された入力が,格子状チートセットを挿入したことでノイズと判定されなくなった現象が確認された.これはDBSCANの性質上,サンプルの評価順序が影響したものと推測する.従って,この現象をチート判定に利用できる可能性は考えにくい.

\subsubsection{DBSCAN以外のアルゴリズムの検討}
実験結果から,クラスタ数はアルゴリズムのパラメータやチートの分布に大きく影響されることが分かる.このため,$k$-平均法\cite{kmean}などのクラスタ数を事前に指定する必要のあるアルゴリズムは適用すべきでない.


\newpage
\part{おわりに}
\section{議論}
\subsection{他への適用}
本研究では「ちくわ栽培キット」およびDjango Webフレームワークについて実験を行ったが,それぞれに特有の条件を用いていない.以下の条件を満たせば他のゲームにも同様の手法が適用できる.

\begin{itemize}
\item
サーバーサイドでDBSCANメソッドが利用可能である.
\footnote{機械学習ライブラリは様々な言語で提供されているため,Python特有の条件ではない.例えばJava向けにWeka\cite{weka}というライブラリが存在し,DBSCANを利用可能である.}
今回はPython用の既存ライブラリであるScikit-learn\cite{sklearn}を用いた.
\item
DBSCANメソッドがHTTPリクエストオブジェクトを取得できる.
\item
サーバーがゲーム進行状態を含むユーザープロフィールを取得できる.
\end{itemize}

\subsection{チートセットの妥当性}
今回はチートセットとして私自身が4通りの格子状チートセットを用意した.当初の予定では,実験協力者にチートも行ってもらうことになっており,そのためのアプリケーションも用意した.しかし,サーバーアプリケーションの更新の際にチートを含むログファイルも更新・初期化されてしまい,利用できなくなったという経緯がある.

この環境ではDBSCANの有効性が確かめられたが,現実的な環境であるとは必ずしも言えない.本来なら,実際にチート被害を受けているゲームのアクセスログに対してこれらの実験を行うのが望ましい.

\section{まとめ}
オンラインゲームにおいては,サーバーへのリクエストパラメータがゲームの進行状態の制限を受ける場合が存在する.このため,サーバーへのリクエストオブジェクトが同一でもチートである場合と正常である場合がある.本研究では,ゲームの進行状態に対応しないリクエストを詐称チートと定義し,この検出を目的とした.

ゲームの進行状態と実際のリクエストを結びつけた情報を成長過程座標と定義し,これに対して
DBSCANによる教師なし機械学習を行うことで詐称チートが検出できると
仮定した.この仮説を検証するために,Androidゲーム「ちくわ栽培キット」を利用した実験を行った.
「ちくわ栽培キット」の成長過程座標のサンプルを取得するために,サーバーアプリケーション上にゲームの進行状態であるゲーム歴を含むログファイルを保存するメソッドを実装した.

「ちくわ栽培キット」のログファイルを成長過程座標に変換し,DBSCANによるクラスタリング解析を行った.その結果,チート利用者にとって利点が大きいアクセスを検出することに成功した.一方で,詐称チートが正常値クラスタの中に属していたり,あまりにチートが多い場合には適用できないことも分かった.

\section{貢献}
オンラインゲームへのチートの一種として詐称チートを定義し,詐称チートの検出方法としてDBSCANによるクラスタリングが有効であることを実験を通して示した.

\section{今後の課題}
この研究ではゲームの進行状態としてゲーム歴のみを用いたが,これは非常に素朴な条件である.一時接続ゲームではサーバーが取得できるクライアント上の情報は限られるが,その中でクライアントの情報をゲーム進行状態として利用することでより精度の高い検出が可能になると期待できる.ゲーム歴以外のゲーム進行状態を用いた詐称チート検出を今後の課題とする.

今回は実験対象が1つのメソッドだけだった.このため今後,他のゲームにもDBSCANが有効であるかどうか,また他のゲームに適用する(アクセスログを取得し保存する,解析を行いチート利用者を特定する)ためにはどうすればよいか確かめる必要がある.

また,オンラインゲームのアクセスログに対するクラスタリング解析は,それ自体が興味深い話題でもある.他のゲームやメソッドと比較して,DBSCANのパラメータ$\Eps, \MinPts$がどう変化するか,クラスタ数はどうなるか確かめる必要がある.これも今後の課題である.

\newpage
\part*{謝辞}
本研究を進めるにあたり,指導教官の権藤克彦先生から丁寧かつ熱心な指導を頂きました.ここに感謝の意を示します.また実験にあたり,会社の製品を快く提供してくださった株式会社アドバンスドテクノロジー様と,貴重なお時間を割いて「ちくわ栽培キット」のアクセスログの収集に協力してくださった権藤研究室の皆さん,およびアドバンスドテクノロジーの社員の皆さんに深く感謝いたします.
 % おわりに・謝辞

\newpage
\begin{thebibliography}{99}

\bibitem{mining}
Lee, Wenke, Salvatore J. Stolfo, and Kui W. Mok. "A data mining framework for building intrusion detection models." Security and Privacy, 1999. Proceedings of the 1999 IEEE Symposium on. IEEE, 1999.

\bibitem{ripper}
Cohen, William W. "Fast effective rule induction." ICML. Vol. 95. 1995.

\bibitem{cluster}
Portnoy, Leonid, Eleazar Eskin, and Sal Stolfo. "Intrusion detection with unlabeled data using clustering." In Proceedings of ACM CSS Workshop on Data Mining Applied to Security (DMSA-2001. 2001.

\bibitem{httplearn}
Kruegel, Christopher, Giovanni Vigna, and William Robertson. "A multi-model approach to the detection of web-based attacks." Computer Networks 48.5 (2005): 717-738.

\bibitem{botcraft}
Gianvecchio, Steven, et al. "Battle of Botcraft: fighting bots in online games with human observational proofs." Proceedings of the 16th ACM conference on Computer and communications security. ACM, 2009.

\bibitem{gamep2p}
Xiang-bin, Shi, et al. "A cheating detection mechanism based on fuzzy reputation management of P2P MMOGs." Software Engineering, Artificial Intelligence, Networking, and Parallel/Distributed Computing, 2007. SNPD 2007. Eighth ACIS International Conference on. Vol. 2. IEEE, 2007.

\bibitem{gamefps}
Laurens, Peter, et al. "A novel approach to the detection of cheating in multiplayer online games." Engineering Complex Computer Systems, 2007. 12th IEEE International Conference on. IEEE, 2007.

\bibitem{ghostrouter}
GHOSTROUTER. http://www.internal.co.jp/products/util/ghostrouter/about/

\bibitem{django}
Django project. https://www.djangoproject.com

\bibitem{python}
Python programming language. http://www.python.org

\bibitem{clf}
Log format. Apache. http://httpd.apache.org/docs/2.4/mod/mod\_log\_config.html

\bibitem{pickle}
Pickle. Python. http://docs.python.jp/2/library/pickle.html

\bibitem{dbscan}
Ester, Martin, et al. "A density-based algorithm for discovering clusters in large spatial databases with noise." KDD. Vol. 96. 1996.

\bibitem{cheatclass}
Yan, Jeff, and Brian Randell. "A systematic classification of cheating in online games." Proceedings of 4th ACM SIGCOMM workshop on Network and system support for games. ACM, 2005.

\bibitem{kmean}
Hartigan, John A. Clustering algorithms. John Wiley \& Sons, Inc., 1975.

\bibitem{sklearn}
Pedregosa, Fabian, et al. "Scikit-learn: Machine learning in Python." The Journal of Machine Learning Research 12 (2011): 2825-2830.

\bibitem{weka}
Hall, Mark, et al. "The WEKA data mining software: an update." ACM SIGKDD Explorations Newsletter 11.1 (2009): 10-18.

\end{thebibliography}


\newpage
\part*{付録}

\end{document}
