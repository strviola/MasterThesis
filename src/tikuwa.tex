\section{ちくわ栽培キット}
\subsection{概要}
「ちくわ栽培キット」は,「ちくわ」をメインキャラクターとした育成シミュレーションゲームである.このゲームは「栽培」「設備投資」「バトル」の要素からなる\footnote{この他に,収穫したちくわの詳細情報を閲覧できる「図鑑」機能がある.}.

\subsubsection{用語の解説}
\paragraph{ちくわ}
ゲーム中で取得可能なキャラクターである.このゲームは「ちくわ」を収集することを最大の目的とする.60種類\footnote{2014年1月現在.今後増える可能性がある.}存在し,それぞれ特徴が異なる.取得したちくわは一時的に「収穫箱」に保存される.

\paragraph{収穫箱}
取得したちくわを保存する機能.収穫箱に存在するちくわに対して,さらに操作が可能である.

\paragraph{ちくわポイント(TP)}
ゲーム内通貨.

\subsubsection{操作の説明}
\paragraph{栽培・収穫}
ちくわを生成する.出現したちくわは「収穫」することで収穫箱に保存される.

\paragraph{売却}
収穫箱にあるちくわを,ちくわごとに定義されたTPと交換する.

\paragraph{設備投資}
TPを消費し,発生するちくわの種類など,栽培に関するパラメータを変更(向上)できる.

\paragraph{バトル}
他人と対戦する.その際,以下の手続きを経る.

\begin{enumerate}
\item
「対戦」コマンドを選択する.この際サーバーとの通信が発生し,収穫箱にあるちくわの種類(ID: 整数)と数(整数)が全て送信される.
\item
プレイヤーは自分が操作するちくわを選択する.「自軍」と呼ぶ.
\item
サーバーに保存されているちくわのリスト(所持者のユーザーID,ちくわID,ちくわの数)から一部を取得する.
\item
プレイヤーはリストから対戦相手とするちくわを選択する.「敵軍」と呼ぶ.
\item
バトルが開始する.プレイヤーは一定時間以内画面をタップすれば勝ち,出来なければ負けである.その時間は自軍と敵軍のパラメータから計算される.
\item
勝敗に応じてイベントが発生する.
\begin{itemize}
\item
プレイヤーが勝った場合,敵軍のちくわのコピーがプレイヤーの収穫箱に追加される.この時,敵軍のちくわIDと数がサーバーに送信される.
\item
プレイヤーが負けた場合,自軍のちくわが収穫箱から削除される.この時,敵軍のちくわの所持者に報酬が送信される.報酬は敵軍のちくわの売却TPと同一である.
\end{itemize}
\end{enumerate}


\subsection{実装}
\subsubsection{クライアントサイド}
クライアントサイドの実装はAndroidで行った.バトル機能を利用する際にHTTP通信を利用する.

\subsubsection{サーバーサイド}
サーバーサイドの実装にはDjango\cite{django}を用いた.DjangoはPython\cite{python}で実装されたWebアプリケーションフレームワークである.Djangoを用いたWebアプリケーション開発では,以下の要素を定義する必要がある\footnote{クライアントがWebアプリケーションである場合,この他にテンプレートHTMLを定義する.今回の「ちくわ栽培キット」のクライアントはAndroidであるため利用していない.}.

\begin{itemize}
\item
モデル.オブジェクト指向言語におけるクラス,データベースにおけるテーブルに該当する.
\item
ビュー関数.リクエストオブジェクトを受け取り,データベースを利用した操作を行い,レスポンスオブジェクトを返す.
\item
URLディスパッチャ.URLとビュー関数を対応づける.
\end{itemize}

\subsection{このゲームで考えられるチート}
\paragraph{過剰ちくわID}
ゲームの進行状態によって,取得可能なちくわIDには制限が存在する.制限を解放するには,設備投資を行って発生ちくわIDの上限を上げる必要がある.しかし,不正アクセスにより制限外のIDをサーバーに送信することが可能である.

\paragraph{過剰ちくわ数}
収穫箱の容量には上限が存在する\footnote{2014年1月現在,この上限は200で固定である.ゲームに課金することでこの上限を拡張する機能を実装予定である.}.従ってサーバーに送信される「ちくわ数」にも制限が存在する.しかし,ここで不正アクセスにより極端に多いちくわ数を送信することも可能である.

\paragraph{過剰報酬}
プレイヤーがバトルに負けた場合,敵軍の所持者に報酬としてTPが送信される.この報酬TPは敵軍のちくわのIDと数によって決まる.しかし,不正アクセスを行うと任意の数を送信することが可能である.

