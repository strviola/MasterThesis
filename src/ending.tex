\part{おわりに}
\section{議論}
\subsection{他への適用}
本研究では「ちくわ栽培キット」およびDjango Webフレームワークについて実験を行ったが,それぞれに特有の条件を用いていない.以下の条件を満たせば他のゲームにも同様の手法が適用できる.

\begin{itemize}
\item
サーバーサイドでDBSCANメソッドが利用可能である.
\footnote{機械学習ライブラリは様々な言語で提供されているため,Python特有の条件ではない.例えばJava向けにWeka\cite{weka}というライブラリが存在し,DBSCANを利用可能である.}
今回はPython用の既存ライブラリであるScikit-learn\cite{sklearn}を用いた.
\item
DBSCANメソッドがHTTPリクエストオブジェクトを取得できる.
\item
サーバーがゲーム進行状態を含むユーザープロフィールを取得できる.
\end{itemize}

\subsection{チートセットの妥当性}
今回はチートセットとして私自身が4通りの格子状チートセットを用意した.当初の予定では,実験協力者にチートも行ってもらうことになっており,そのためのアプリケーションも用意した.しかし,サーバーアプリケーションの更新の際にチートを含むログファイルも更新・初期化されてしまい,利用できなくなったという経緯がある.

この環境ではDBSCANの有効性が確かめられたが,現実的な環境であるとは必ずしも言えない.本来なら,実際にチート被害を受けているゲームのアクセスログに対してこれらの実験を行うのが望ましい.

\section{まとめ}
オンラインゲームにおいては,サーバーへのリクエストパラメータがゲームの進行状態の制限を受ける場合が存在する.このため,サーバーへのリクエストオブジェクトが同一でもチートである場合と正常である場合がある.本研究では,ゲームの進行状態に対応しないリクエストを詐称チートと定義し,この検出を目的とした.

ゲームの進行状態と実際のリクエストを結びつけた情報を成長過程座標と定義し,これに対して
DBSCANによる教師なし機械学習を行うことで詐称チートが検出できると
仮定した.この仮説を検証するために,Androidゲーム「ちくわ栽培キット」を利用した実験を行った.
「ちくわ栽培キット」の成長過程座標のサンプルを取得するために,サーバーアプリケーション上にゲームの進行状態であるゲーム歴を含むログファイルを保存するメソッドを実装した.

「ちくわ栽培キット」のログファイルを成長過程座標に変換し,DBSCANによるクラスタリング解析を行った.その結果,チート利用者にとって利点が大きいアクセスを検出することに成功した.一方で,詐称チートが正常値クラスタの中に属していたり,あまりにチートが多い場合には適用できないことも分かった.

\section{貢献}
オンラインゲームへのチートの一種として詐称チートを定義し,詐称チートの検出方法としてDBSCANによるクラスタリングが有効であることを実験を通して示した.

\section{今後の課題}
この研究ではゲームの進行状態としてゲーム歴のみを用いたが,これは非常に素朴な条件である.一時接続ゲームではサーバーが取得できるクライアント上の情報は限られるが,その中でクライアントの情報をゲーム進行状態として利用することでより精度の高い検出が可能になると期待できる.ゲーム歴以外のゲーム進行状態を用いた詐称チート検出を今後の課題とする.

今回は実験対象が1つのメソッドだけだった.このため今後,他のゲームにもDBSCANが有効であるかどうか,また他のゲームに適用する(アクセスログを取得し保存する,解析を行いチート利用者を特定する)ためにはどうすればよいか確かめる必要がある.

また,オンラインゲームのアクセスログに対するクラスタリング解析は,それ自体が興味深い話題でもある.他のゲームやメソッドと比較して,DBSCANのパラメータ$\Eps, \MinPts$がどう変化するか,クラスタ数はどうなるか確かめる必要がある.これも今後の課題である.

\newpage
\part*{謝辞}
本研究を進めるにあたり,指導教官の権藤克彦先生から丁寧かつ熱心な指導を頂きました.ここに感謝の意を示します.また実験にあたり,会社の製品を快く提供してくださった株式会社アドバンスドテクノロジー様と,貴重なお時間を割いて「ちくわ栽培キット」のアクセスログの収集に協力してくださった権藤研究室の皆さん,およびアドバンスドテクノロジーの社員の皆さんに深く感謝いたします.
