\section{実験環境}
\subsection{プログラム動作環境}
\subsubsection{クライアントサイド}
\begin{itemize}
\item
プラットフォーム: Android 3.0 (API11 HONEYCOMB) $\sim$ 4.4 (API19 KITKAT)
\item
端末: ICONIA Tab, Galaxy Nexus, Arrows Z, MEDIAS Tab, 等.
\end{itemize}

\subsubsection{サーバーサイド}
\begin{itemize}
\item
OS: Ubuntu
\item
サーバーソフトウェア: Apache Server
\end{itemize}

\subsection{正常入力の収集}
今回は「ちくわ栽培キット」における「収穫箱の情報をアップロードする」メソッドに注目した.
この実験における成長過程座標は以下の通りである.

\begin{center}
{\tt [time, tid, num]}
\end{center}

ただし,

\begin{itemize}
\item
{\tt time}: 「ちくわ栽培キット」のゲーム暦
\item
{\tt tid}: 送信されたちくわのID.
\item
{\tt num}: IDに対応するちくわの所持数.
\end{itemize}

である.なお,{\tt tid, num}共にゲームの進行上の制限を受ける成長パラメータである.

このメソッドのサンプルを収集するため,
学内および株式会社アドバンスドテクノロジーの従業員に協力を呼びかけ,計10人に「ちくわ栽培キット」を遊んでもらった.この結果,2013年12月24日から同月30日の期間に2128件の正常入力サンプルを集めることができた.

\subsection{予備実験: 正常入力のクラスタ解析}
チートを判定する前に,正常入力がノイズのないクラスタと判定されるかどうか,またその時のパラメータ$\Eps, \MinPts$の値はいくらか確かめる必要がある.なお,正確な外れ値判定を行うためには,$\Eps$はなるべく小さく,$\MinPts$はなるべく大きい方が望ましい.

\paragraph{$\MinPts$の影響}
$\Eps = 0.5$に固定し,$\MinPts$を次々に変化\footnote{$\MinPts = 1$だとノイズが判定できないため意味が無い.}させた時のクラスタ数(\#C),ノイズ数(\#N)は次の通りである.なお詳しくは後述するが,$\Eps = 0.5$はかなり厳しい(ノイズが発生しやすい)条件である.

\begin{table}[htbp]
\begin{center}
\begin{tabular}{c|cc}
$\MinPts$ & \#C & \#N \\ \hline
2 & 13 & 0 \\
3〜4 & 12 & 4 \\
5〜10 & 11 & 8 \\
11 & 10 & 27
\end{tabular}
\end{center}
\end{table}

この結果から,正常値サンプルのうち4点は他の位置から離れたクラスタを形成し,その他の4点がさらに離れた位置で別のクラスタを形成していると推測できる.以後の実験では,$\MinPts=2, 4, 10$の時に注目する.

\paragraph{$\Eps$の影響}
入力を正常サンプルのみとし,$\MinPts = 10$に固定して$\Eps$を変化させた時のクラスタリング結果は以下の通りである.

\begin{table}[htbp]
\begin{center}
\begin{tabular}{c|cc}
$\Eps$ & \#C & \#N \\ \hline
0.4 & 14 & 53 \\
0.5 & 10 & 27 \\
0.6 & 8 & 8 \\
0.7〜0.9 & 5 & 8 \\
1.0〜1.4 & 5 & 2 \\
1.5 & 3 & 0
\end{tabular}
\end{center}
\end{table}

この結果からも,サンプルのうち8点は他のクラスタから離れていることが推測できる.また,$\Eps=0.6, 1.0, 1.5$の時にそれぞれ大きな変化があることも分かる.以後の実験では,この3値に注目する.

\subsection{チートの挿入}
この実験では成長パラメータが2個存在するため,その2個に関する格子状チートセットを用意した.また,解析アルゴリズムであるDBSCANは点と点の間隔に強く影響されるため,格子の間隔は以下の4通りとした.なお,「ゲーム初心者がチートを使ってゲーム終盤でしか手に入らないアイテムを入手した」場合を想定しているため,挿入したチートのゲーム歴は正常入力のそれに比べて無視できる値である.このため,以降チートセットのゲーム歴は0とし,表記を省略する.

\begin{table}[htbp]
\begin{center}
\begin{tabular}{c|ccc}
& 総チート数 & {\tt tid}間隔 & {\tt num}間隔 \\
\hline
Sparse & 15 & 14 & 5000 \\
Mid & 45 & 7 & 1000 \\
Densely & 200 & 3 & 100 \\
Hyper-densely & 200 & 1 & 10
\end{tabular}
\end{center}
\end{table}

\section{結果}
\subsection{攻撃の分布の影響}
予備実験の結果を踏まえ,$\Eps = 0.6, \MinPts = 10$として各チートセットを混ぜたサンプルにDBSCAN解析を行い,False Positive (\#FP), False Negative (\#FN)を測定した.結果は以下の通りである.

\begin{table}[htbp]
\begin{center}
\begin{tabular}{c|ccc}
チートセット & \#C & \#FP & \#FN (rate) \\ \hline
Sparse & 6 & 0 & 1 (6.7\%) \\
Mid & 6 & 0 & 2 (4.4\%) \\
Densely & 5 & 0 & 11 (5.5\%) \\
Hyper-densely & 6 & 8 & 200 (100\%)
\end{tabular}
\end{center}
\end{table}

この結果から,チートが少ない場合はDBSCANによりチートが検出できることが分かる.ただし,チートそのものの密度が正常入力と区別できなくなると,この方法は使えなくなる.

なお,各チートセットで検出できなかった(False Negativeであった)チート入力は以下の通り.

\begin{table}
\begin{center}
\begin{tabular}{l|l}
Sparse & $[0, 1]$ \\ \hline
Mid & $[0, 0], [0, 7]$ \\ \hline
Densely & $[0, 0], [0, 100], [0, 200], [3, 0], [3, 100], [3, 200],$ \\
& $[6, 0], [6, 100], [9, 0], [9, 100], [12, 0]$
\end{tabular}
\end{center}
\end{table}

ゲーム歴が0であることを考えると,この状態で自然な入力は{\tt tid, num}共に小さいものである.従って,DBSCANも自然な入力に近いものに対して弱いといえる.

\subsection{アルゴリズムのパラメータの影響}
実際のゲームの運用環境においては,チートの数は正常入力の数より充分小さいと推測される.すなわち,これらのチートセットのうち最も現実に即しているのはSparseである.
これを踏まえ,Sparseチートセットに対して$\MinPts=10$とし$\Eps$を変更しながら解析を行った.結果は以下の通りである.

\begin{table}[htbp]
\begin{center}
\begin{tabular}{c|ccc}
$\Eps$ & \#C & \#FP & \#FN \\ \hline
0.1 & 34 & 34 & 1 \\
0.2 & 25 & 12 & 1 \\
0.3 & 6 & 0 & 1 \\
$\vdots$ & \multicolumn{3}{c}{$\vdots$} \\
2.2 & 2 & 0 & 1 \\
2.3 & 2 & 0 & 2
\end{tabular}
\end{center}
\end{table}

$\Eps=2.3$の時点でFalse Negativeが増加した.この時検出できなかったチートサンプルは$[14, 1]$である.
予備実験において$\Eps=1.5$でノイズが出なかったことを踏まえると,正常値セットに対してノイズが出ない$\Eps, \MinPts$の値はチート検出にも有効であると結論できる.

\subsection{考察}
\subsubsection{検出失敗の理由}
この実験では,どのチートセットに対してもFalse Negativeが発生した.これは,チートの密度が低い場合は正常入力と近い値が検出できなかったことに起因する.チート利用者にとって,正常入力に近い入力をわざわざチート経由で行う利点はない\footnote{Hyper-denselyチートセットに匹敵する量のチートが送信された場合はこの限りではない.ただし,オンラインゲームの実運用でこれが発生するようなら,サーバーアプリケーションのセキュリティを根本的に見直した方がいいだろう.}.従って,DBSCANは詐称チート検出に効果があると結論できる.

\subsubsection{パラメータの見積もり}
$\MinPts$を小さくすると,チート同士でクラスタを形成し,ノイズと判定されなくなる可能性が高くなる.また$\Eps$を小さくすると逆に正常な入力がノイズと判定される確率が上がる.これらの値は正常値データセット(もしくはなるべく正常値の割合が高いデータセット)に対して予備実験を行い,ノイズが出ない値に調整する必要がある.

\paragraph{チートとノイズの関係}
この実験で,予備実験では正常入力であるにも関わらずノイズと判定された入力が,格子状チートセットを挿入したことでノイズと判定されなくなった現象が確認された.これはDBSCANの性質上,サンプルの評価順序が影響したものと推測する.従って,この現象をチート判定に利用できる可能性は考えにくい.

\subsubsection{DBSCAN以外のアルゴリズムの検討}
実験結果から,クラスタ数はアルゴリズムのパラメータやチートの分布に大きく影響されることが分かる.このため,$k$-平均法\cite{kmean}などのクラスタ数を事前に指定する必要のあるアルゴリズムは適用すべきでない.
