\part{はじめに}
\section{背景}
\subsection{不正行為の検出}
ネットワークを用いたアプリケーションは広くアクセス可能であるという利点がある反面,しばしば脆弱性を悪用した攻撃の対象になりうる.
Webアプリケーションの利用の一つとしてオンラインゲームが存在するが,これも攻撃の対象として例外ではない.
オンラインゲームにおける不正行為は特にチートと呼ばれ,チート検出に関して数多くの既存研究が存在する.
過去のチート検出に関する研究のほとんどはMMORPGやFPSを対象としており,チートの内容もゲーム内のキャラクターの移動に関するものである.
このため,過去の研究ではサーバーへの入力のみに注目しており,実際にそれで充分であった.
例えば\cite{botcraft}では代表的なMMORPGである ``World of Warcraft'' における,プログラムによる自動操作キャラクター(Bot)の検出を目的としている.\cite{gamefps}はオンラインFPSにおけるキャラクターの操作経路を取得し,これを正常なプレイヤーの操作経路と比較するというアプローチをとっている.

\subsection{一時接続ゲーム}
近年,ゲーム上の入力の全てではなく,その中の一部だけ,もしくはそれを加工したデータをサーバーに送信する設計のオンラインゲームが普及しつつある.
本論文ではこの特徴を「一時接続」と呼ぶ.
一時接続ゲームは以下の理由から,特に携帯端末プラットフォームにおいて顕著である.

\begin{itemize}
\item
消費電力を抑える.携帯端末はバッテリーが有限であり,さらにゲーム以外のアプリケーションにも影響するため.
\item
屋外での利用も想定すると,電波環境が安定しない.
\end{itemize}

一時接続ゲームにもチートは存在しうる.より一般的に,以下の条件を満たすサーバー・クライアント型アプリケーションにはチートが発生する.

\begin{enumerate}
\item
サーバーにデータを送信し,他の人(アカウント)と共有する
\item
クライアントの設計上、送信データに「制約」が存在する
\item
制約外のデータ(「チート」と呼ぶ)を送信することで、他人の行動や意思決定に影響する
\item
チートを行うことで自分に有利になる
\end{enumerate}

ここで,この条件が妥当である理由を,具体例を交えて説明する.
まず1は最大の前提である.
そもそもインターネットに繋がっていないゲームであれば,たとえ改造者がクライアントを改造するなどしてプレイヤーに不正に有利な状態を作ったとしても何ら問題はない.
それは改造者が勝手に遊んでいるだけで,他人に迷惑をかけていない.
\footnote{二次創作物やそれに対する著作権などの問題が発生することが考えられるが,今回の研究の興味の対象ではない.}
また2に関して,「制約」が存在する全てのアプリケーションでチートが問題になるとは限らない.
例えばTwitterは「140文字以上の文字列を送信できない」という「制約」が存在するが,この制約を守らないチート的な送信があったところでそれは他人に何ら影響はない.

問題は3の「他人に影響する」という特徴である.
オンラインゲームは遠い所にいる他人との同時プレイを可能にするものであり,それが醍醐味のひとつでもある.
一方でそれを可能にしている技術には欠陥が含まれることも多く,チートを許してしまうケースも存在する.
例えば,ガンホー社の「パズル\&ドラゴンズ」(パズドラ)では,外部ツール\cite{ghostrouter}を利用してゲーム内のデータを任意の数値に書き換えるチートが可能である.
このチートを行った時,プレイヤー本人が有利な状態を不正に作る
\footnote{強力なキャラクターを使用できる,既存のキャラクターの能力を向上させる,などが可能である.}
だけでなく,他人の意思決定に影響してゲーム内ポイントをより多く獲得することも期待できる.

\section{問題点}
\subsection{詐称チート}
一時接続ゲームにおけるチートの多くは,様々なパラメータを偽り,チート利用者に有利な状態を作ることを目的とする.
この操作は,ゲーム内のプロセスを経ることなく,キャラクターやアイテムを使用可能にし,またキャラクターの性能を上げるなどの効果をもたらす.
本研究ではこのチートに注目し,「詐称チート」と定義する.
既存のチート検出システムは,詐称チートを検出できるかどうか自明ではない.これはサーバーへの入力(または入力の集合)が同一であっても,チートと見なすべき場合と正常と見なすべき場合が存在するためである.

具体的な例で考える.RPGにおいて,アイテムAはゲーム終盤のダンジョンXでのみ手に入るという環境を想定し,「プレイヤーPがAを取得した」というリクエストRがサーバーに送信されたとする.
もしRが正常なリクエストであれば,Pは実際にゲーム内のイベントを経由し,ダンジョンXに到達したはずである.
一方Rが詐称チートであれば,Pはゲームのプログラムを不正に加工し,リクエストのみを送信したことになる.
ここで,チートの判定のために単純な条件式を用いることは得策ではない.
「PはXに入るためのイベントYを達成し,アイテムZを持っていて,経験値がN以上であり…」と,ゲームの設計によってその判定式・フラグはいくらでも複雑になりうる.

\section{研究の目的}
本研究では,一時接続ゲームにおけるクライアント設計に非依存な詐称チートの検出を目的とする.すなわち,ゲームの設計を含むクライアントの情報を得られない環境で,詐称チートをサーバー上で検出する方法を提案する.

\section{アプローチ}
\subsection{成長過程座標}
本研究では,ユーザーごとの「ゲームの進行状態」に注目した.これはユーザーが自分で操作できない数値である.また,ゲームの進行状態に影響されるリクエストパラメータを「成長パラメータ」と定義し,ゲーム進行状態と成長パラメータを連結した数値リストを「成長過程座標」としてこれに注目した.

成長過程座標には,詐称チートと正常入力の差が生じることが期待できる.例えば先のRPGの例において,ゲームの進行状態としてゲームのプレイ時間$t$,成長パラメータとしてアイテムのID$i$をとれば,成長過程座標$(t, i)$が作成できる.ここで,一般的なプレイヤーはアイテムAの取得に50時間かかるとする.すなわち多くのは$(50, A)$前後である.この時,プレイヤーPから作成された成長過程座標が$(10, A)$であれば,Pはチートを行っている可能性が高い.

\subsection{クラスタリングによる詐称チートの検出}
成長過程座標に関しても,何がチートに該当するかはゲームの設計に依存しており,単純な条件式で判別すべきでない.本研究ではクラスタリングを用いることで,ゲームの設計に非依存な判定が可能であることを示す.すなわち,
大多数の正常な入力を「正常値クラスタ」,一部のチート入力を「ノイズ」として判定することで,詐称チートを判定する.既に複数のクラスタリングアルゴリズムが知られているが,本研究ではその中のDBSCANを採用した.DBSCANはクラスタ数を事前に見積もる必要がなく,またどのクラスタにも属さないノイズを検出できるため,詐称チートの検出に適していると考えたためである.

\section{実験}
\subsection{ちくわ栽培キット}
このアプローチを検証するために,実際のオンラインゲームにおける多数の正常値入力セットを用意する必要がある.本研究では株式会社アドバンスドテクノロジー様の協力により,Android向けソーシャルゲーム「ちくわ栽培キット」のサーバーアクセスログの提供を受けた.

\subsection{結果}
提供された2128件の正常値サンプルと自分で用意したチートセットを利用し,DBSCANを用いたチート判定を行った.その結果,チートの数が正常値の数と比べて充分に少なく,かつチートによりチート利用者が極端に有利になる場合においては,クラスタリングがチート判定方法として有効であると示した.

\section{貢献}
既存のチート検出器では対応できないチートとして,一時接続ゲームにおける詐称チートを定義した.詐称チートに対して,ゲーム進行状態と成長パラメータを組み合わせた成長過程座標に対するクラスタリングが検出方法として有効であることを,実験を通して示した.


\section{関連研究}
本研究はオンラインゲームのチート検出を目標としているため,まずこの分野の先行研究について述べる.次に,機械学習を用いた不正入力の検出に関連して侵入検知システム(Intrusion Detection System, IDS)を取り上げる.

\subsection{オンラインゲームのチート検出}
Gianveccioら\cite{botcraft}はMMOGのひとつである「World of Warcraft」におけるBot(自動操縦プログラム)を観測的に検出する方法を提案している.この研究は,操作キャラクターのキーボードおよびマウスの操作をリアルタイムに観測し,異常に正確もしくは素早い操作を行っているキャラクターを検出することで,不正行為を発見する手法を提案した.Laurensら\cite{gamefps}はチートを許してしまう技術的もしくは開発プロセス上の欠陥について調査ている.また,それらに対する防御策として,オンラインFPSにおけるWall-hack cheatingを観測的に検出するシステムを提案している.

Shiら\cite{gamep2p}はP2Pアーキテクチャを採用したMMOGにおいて,各プレイヤーの操作から数値的に信頼度を計算することでチート検出を試みている.ここでいう「信頼度」の計算には,同様にプレイヤーの逐次的な行動(キャラクターの動く速度,その正確さ,など)が必要であるため,一部のデータのみをサーバー経由で共有する一時接続ゲームには適用できない. 

\subsection{IDS}
Portnoyら\cite{cluster}はTCPアクセスログにクラスタリングを用いることで,DOS攻撃やR2L攻撃の教師なし侵入検知を行った.Kruegelら\cite{httplearn}はHTTPリクエストパラメータに対して,複数の統計処理や自然言語解析に基づいた解析手法を適用することで,バッファオーバーフローやXSS(Cross Site Scripting)などの侵入検知を行った.
Leeら\cite{mining}はデータマイニングアルゴリズムであるRIPPER\cite{ripper}を用いることで,データを解析し拡張性の高い侵入検知システムを構築する研究を行っている.
