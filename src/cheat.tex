\section{チートの定義}
この節では,本研究の興味の対象とするチートの定式化と,それを検出するための概念の定義を行う.

\subsection{詐称チート}
一般に,オンラインゲームにおいて既に多種多様のチートが報告されている\cite{cheatclass}.本研究では,その中でも「間違った信用を利用したチート」,すなわちクライアントサイドのプログラムを不正に改造することによるチートを興味の対象とする.

「ちくわ」の例では,クライアントアプリケーションのソースコードの漏洩
\footnote{もちろん,あまり簡単にサーバーに侵入できないような対策をとっているが,いずれも本質的にアプリケーション以外からのアクセスを防ぐものではない.}
などにより,ゲームの状態を無視した不正なHTTPリクエストをサーバーに送信することが可能である.本研究ではこの概念を定式化した{\bf 詐称入力}および{\bf 詐称チート}を以下の通り定義する.

\begin{mathdef}[詐称入力]
サーバー・クライアント型アプリケーションで,クライアントアプリケーションのソースコード以外に由来するサーバーへのHTTPリクエストを詐称入力と呼ぶ.
\end{mathdef}

\begin{mathdef}[詐称チート]
詐称入力のうち,クライアントアプリケーションが以下の条件を満たす時,それを詐称チートと呼ぶ.

\begin{enumerate}
\item
クライアントの設計上、送信データに「制約」が存在する.
\item
詐称入力を送信することで、他人の行動や意思決定に影響する.
\item
詐称入力を行うことで自分に有利になる.
\end{enumerate}

\end{mathdef}

\subsection{成長過程座標を用いた詐称チートの検出}
本研究では,「ちくわ」を例に取って詐称チートを検出することを試みる.詐称チートは以下の理由から,単純な条件で判定できるかどうか自明ではない.

\begin{enumerate}
\item
サーバーからは,詐称入力か通常入力かどうか判別できない.
\item
送信されうるHTTPリクエストパラメータはクライアントアプリケーションの設計に依存する.パラメータの間に依存関係が存在する場合もありうる.
\item \label{cheatnot}
同じHTTPリクエストパラメータでも,詐称チートになる場合とそうでない場合が存在する.
\end{enumerate}

特に理由\ref{cheatnot}はこの問題を難しくしている.例として,「ちくわ」はバトルの際に収穫箱の情報をサーバーに送信する.この時のHTTPリクエストパラメータは「ちくわ毎に割り当てられたID」と「そのIDのちくわのその時点での所持数」である.ここで,各時点における取得可能なちくわ(ID)には制限が存在し,その制限は設備投資を行うことで拡張できる.このため,設備制限外の入力は詐称チートと見なすのが自然であるが,設備制限の情報をサーバーが取得するのはコストがかかるため得策ではない
\footnote{仮にサーバー側で設備制限情報を用いた判定メソッドを実装する場合,同じ動作をするソースコードをサーバーとクライアントで2重に用意する必要がある.また,その判定メソッドは汎用性がなく,ビュー関数ごとに用意しなければならない}.

\paragraph{アプローチ: 成長過程座標}
詐称チートが発生するのは,そのパラメータにアプリケーションの設計上の制約が存在するためである.本研究では以下の条件を満たすパラメータを特に{\bf 成長パラメータ}として区別する.

\begin{enumerate}
\item
ゲームの設計上,パラメータが取りうる値に制約が存在する.
\item
ゲームの進行に従って,その制約が拡張される.
\end{enumerate}

成長パラメータと,一般的に取得可能なゲームの進行状態を組み合わせれば,詐称チートが検出できることが期待できる.本研究ではこの目的を達成するために{\bf 成長過程座標}を定義し,ゲームの進行状態として{\bf ゲーム歴}を採用した.

\begin{mathdef}[成長過程座標]
ゲームの進行状態を表す数値(複数可)を$p_1, \dots, p_m (m \geq 1)$,成長パラメータを$q_1, \dots, q_n (n \geq 1)$とする.この時,ベクトル \[
(p_1, \dots, p_m, q_1, \dots, q_n)
\] を成長過程座標とする.
\end{mathdef}

\begin{mathdef}[ゲーム歴]
ユーザーがHTTPリクエストを送信した日時とユーザーがWebサービスに会員登録した日時の差の秒数をゲーム歴とする.
\end{mathdef}
